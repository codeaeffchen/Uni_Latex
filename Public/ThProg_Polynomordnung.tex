	\documentclass{article}
	\usepackage{amsmath}
	\usepackage{nccmath}
	\usepackage{ulem}
	\DeclareMathSizes{10}{10}{10}{10}
	\setlength{\parindent}{0pt}
	\title{Ko-Rekursion/-Induktion}
	\date{Oktober 2015}
	\begin{document}
	\maketitle
	\section{In Funktionsschreibweise bringen}
	Infixnotation:
	\[x \uparrow ( y \uparrow z) \rightarrow_{0}\; x \uparrow (y \downarrow y)\]\\
	Funktionsschreibweise:
	\[ P_{\uparrow}(x,P_{\uparrow}(y,z)) \rightarrow_{0} \; P_{\uparrow}(x,P_{\downarrow}(y,y))\]
	\section{Kontext ggf. kuerzen}
	\[ P_{\uparrow}(x,P_{\uparrow}(y,z)) \rightarrow_{0} \; P_{\uparrow}(x,P_{\downarrow}(y,y))\]
	\[\xout{P(x},P_{\uparrow}(y,z)) \rightarrow_{0} \; \xout{P(x},P_{\downarrow}(y,y)))\]
	\[P_{\uparrow}(y,z) \rightarrow_{0} \;P_{\downarrow}(y,y)\]
	\section{Polynom finden}
	\[P_{\uparrow}(y,z) \rightarrow_{0} \;P_{\downarrow}(y,y)\]
	Ein "+" zwischen die Parameter setzen und Multiplikator vor beide sodass gilt:
	\[P_{\uparrow}(y,z) > \;P_{\downarrow}(y,y)\;\;\textbf{\underline{bzw:}}\;\;ay+bx>cy+dy\]
	\textbf{Ansatz:}\\
	- zweite Formel linke Seite ist syntaktisch echt groesser als die Rechte\\
	- erste Formel P$\uparrow$ ist auf beiden Seiten gleich also beliebig \\
	- niemals Minuswerte \\
	- niemals '0' als Multiplikator\\\\
	\textbf{hier:}
		\[ay+bx>cy+dy\]
	\textbf{Wir nehmen an dass wir y hoch genug setzen damit bx irrelevant wird:}\\
		\[ay>cy+dy = a>c+d \]
	\\ \textbf{das ist nun trivial, wir raten:}
	\begin{fleqn}
		\begin{align*}
			&c = 1 \\
			&d = 1 \\
			&a = c+d+1 = 3
		\end{align*}
	\end{fleqn} \\
	\textbf{und damit die Polynome:}\\
	\[P\downarrow = x_1+x_2 \;\;\; und \;\;\; P\uparrow = 3x_1+x_2 \] \\
	\section{Domaenen und Grenzfaelle}
	Unsere Polynome gelten fuer kleine Werte nicht, um genauzusein, nur fuer die 0 nicht, daher geben wir als Domaene an:
	\[ A = N\setminus\{0\} \]
und Funktionsdomaene:\\
	\[\mathcal{A} = \{P(*),P(*)\}\]
	
	
	
	
	
	
	
	
	
	
	
	
	
	
	
	
	
	
	
	
	
	
	
	\end{document}