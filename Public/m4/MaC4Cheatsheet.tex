	\documentclass{article}
	\usepackage{amsmath}
	% -------- Umlaute korrekt ----------------
	\usepackage[utf8]{inputenc}
	\usepackage[ngerman, english]{babel}
	%-------------------------------------------

	% TikZ Library
	\usepackage{tikz}
	\usetikzlibrary{arrows,backgrounds,positioning,fit,calc,petri}
	\usetikzlibrary{shapes, shapes.misc}
	\usetikzlibrary{decorations.markings,decorations.pathmorphing}
	% -----------------------------------------
	% Einrueckung unterbinden nach Absatz
	\setlength{\parindent}{0pt}

	\DeclareMathSizes{10}{10}{10}{10}
	\title{Mathe C4 Merz - Cheatsheet}
	\author{Yannik Schmidt (Sheppy)\\September 2015}
	\date{Diesen Zusammenfassung kann Fehler enthalten!}
	\begin{document}
	\maketitle
	\section{Statistik}
	\subsection{empirisches arithmetisches Mittel}
	\[x_{arith}=\frac{1}{n}\sum_{i=1}^n x_i\]
	\subsection{empirischer Median (Zentralwert)}
	\[
	x_{median}=
		\begin{cases}
			\frac{x_{n+1}}{2}								& \text{n ungerade} \\
			\frac{x_{n/2} \;\; + x_{(n+1)/2}}{2}	& \text{n gerade}
		\end{cases}
	\]
	Wobei der Index fuer die n'te Zahl in einer Angabe in Stile von \{A,B,C,...\} steht.
	\subsection{empirische Varianz}
	\[x_{var}=\frac{1}{n-1}\sum_{i=1}^n (x_i-x_{median})\]
	\subsection{Gleichgewichtsverteilung}
	\[
	G_{var} = 
		\begin{pmatrix}
		1 \\
		. \\
		. \\
		1
		\end{pmatrix}
		*\left [
		\begin{pmatrix}
		1&.&.& 0 \\
		. & 1 &.& . \\
		. & . &1& . \\
		0&.&.&1
			
		\end{pmatrix}-P+
		\begin{pmatrix}
		1&.&.&1 \\
		.&.&.&. \\
		.&.&.&. \\
		1&.&.&1
		\end{pmatrix}\right ]
		^{-1}
	\]
	Wobei P die Uebergangsmatrix ist. Die Alternative ist die Matrix solang zu potentieren bis sie konvergiert.
	\section{Mengen}
	\subsection{o-Algebra}
	- leere Menge enthalten\\
	- alle Kombinationen der Elemente enthalten, die nicht bereits gemeinsamme Elemente haben also z.B. \textbf{NICHT} \{x,y\} und \{y,z\} zu \{x,y,z\} machen\\
	- alle Komplemente enthalten\\ \\
	\textbf{Beispiel:}\\
		Grundmenge = $\{1,2,3,4\}$\\
		NICHT o-Algebra Menge = $\{\{1,2\},\{3\}\}$\\
		o-Algebra Menge = $\{\emptyset ,\{1,2\},\{3\},
		\underbrace{\{1,2,3\}}_{\substack{\{1,2\}\{3\}}},
		\underbrace{\{3,4\}}_{\substack{\neg \{1,2\}}},
		\underbrace{\{4\}}_{\substack{\neg \{1,2,3\}}},
		\{1,2,3,4\},\{1,2,4\}\}$
	
	\section{Wahrscheinlichkeiten}
	\subsection{Wuerfeln}
	\subsubsection{keine 6}
		\[
			p_0 = \left( \frac{5}{6} \right)^n , n = \text{Anzahl der Wuerfe}
		\]
	\subsubsection{mindestens 'x' 6er (Gegenereignis)}
		\[	
			p_1 = 1 - \left( \frac{5}{6} \right)^n = 1 - p_0
		\]	
		\[		
			p_2 = 1-\left(1 - \left( \frac{5}{6} \right)^n\right)-\left( \frac{5}{6} \right)^n = 1-p_1 -p_0
		\]
		\[
			p_x = 1 - \sum_{i=0}^{x-1} p_i
		\]
	\subsubsection{6er-Pasch bei 2 Wuerfeln}
		$Ereignisraum = 6^2 , \text{Anzahl guenstiger Ereignisse = 1 , naehmlich (6,6)}$\\
		dann wieder ueber Gegenereignis: \\
		\[ p=1-\left(\frac{35}{36}\right)^n \]
	\subsubsection{genau eine 6 bei n-Wuerfeln/Wuerfen}
		\[ p= \frac{n*5^{(n-1)}}{6^n}\]\\
		- $6^n $ ist wie immer die Anzahl der Gesamtmoeglichkeiten \\
		- es gibt n-Moglichkeiten an der die 6 sein kann \\
		- es bleiben bei den verbleibenden n-1 Wuerfen 5 Moeglichkeiten
	\subsubsection{genau x-6er bei n-Wuerfeln/Wuerfen}
		\[ p= \frac{\begin{pmatrix}
		x\\n
		\end{pmatrix}5^{(n-x)}}{6^n}\]\\
		\[\begin{pmatrix}
		x\\n
		\end{pmatrix}= \frac{n!}{k!(n-k)!}
		\]\\
		$\textbf{oder noch allgemeiner, mit Anzahl Moeglichkeiten 'z' (z.B. 6 bei Wuerfel):}$\[
		p= \frac{\begin{pmatrix}
		x\\n
		\end{pmatrix}(z-1)^{(n-x)}}{z^n}
		\]
	\subsubsection{Seiten mit verschiedenen Wahrscheinlichkeiten}
		z.B. 6 Seiten mit normaler Wahrscheinlichkeit $(w_1)$, 8 Seiten mit 1/4 Wahrscheinlichkeit
		$(w_2)$, wir exploiten die Tatsache, dass: \\ \[ \sum
		(Teil-)Wahrscheinlichkeiten = 1 \]\\
		also:\\
		\begin{equation}	
			6w_1 + 8w_2 = 1 \end{equation}
		\begin{equation}	
			\frac{1}{4}w_1 = w_2 
		\end{equation}\\
		Zwei Gleichungen, zwei Unbekannte, easy mode.
	
	\section{Bedingte Wahrscheinlichkeiten}
	\subsection{Beispiele}
	\subsubsection{Krankheitstest}
	0,2\% Krank, 95\% der Kranken werden erkannt, 98\% der Gesunden werden richtig erkannt\\ \\
	\textbf{Wie viele als Krank erkannte wirklich krank?}\\
	\[
		P(K | K_{ident} ) = \frac{P(K_{ident}|K)*P(K)}
								{P(K_{ident}|K)*P(K)+P(K_{ident}|K)*P(\neg K)} =
									\frac{0,95*0,002}{0,95*0,002+0,002*0,998} = 8,7\%
	\]
	\subsubsection{min. eine 6 unter Bedingung verschiedene Augenzahlen}
	\[
		P(min. eine 6|verschiedene Augenzahlen) = \frac{\text{Moeglichkeiten verschiedene Augenzahlen UND min. eine 6}}{\text{Moeglichkeiten verschiedene Augenzahlen}}
	\]\\
	\[
		p=\frac{n*(6-1)!-(6-n)!}{6!-n!} 
	\]
	bei 3 Wuerfeln also z.B.:\[
		p=\frac{3*5!-3!}{6!-3!} = \frac{3*5*4}{6*5*4} = 0,5
	\]
	
	\section{Wahrscheinlichkeitsfunktionen}
	\subsection{Eigenschaften von Wahrscheinlichkeitsfunktionen}
	\[ \sum_{w \in \Omega} f(w) = 1 \text{ (die Summe aller Wahrscheinlichkeiten ist 1)}\]
	und logischerweise:
	\[ \forall w\in\Omega . f(w)>=0 \text{ (keine negativen Wahrscheinlichkeiten)} \]
	\subsection{Momenterzeugende Funktion}
	\[
		M(t)=\sum_{n\in\Omega}^{\infty}(e^t)^n * f(n)
	\]
	- f(n) ist die gegebene Wahrscheinlichkeitsfunktion\\
	- 'n' koennte z.B. definiert sein als $n=\{1,2,3,...\}$
	\subsection{Erzeugende Funktion}
	\subsubsection{Wahrscheinlichkeitsfunktion berechnen}
	\subsubsection{Mitterlwert}
	\subsubsection{Varianz}
	TODO
	\subsection{Mittelwert, Varrianz}
	\begin{itemize}
		\item Mittelwert: $m_1 = \sum_{n=0}^\infty n*f(n)$
		\item Varianz: $\widehat{m}_2 = m_2 - m_1^2$
	\end{itemize}
	\section{Verteilungen}
	\subsection{Allgemein}
	\subsubsection{Eigenschaften}
	\begin{itemize}		
		\item stetig
		\item monoton steigend
		\item $\lim_{t \to \infty} G(t) = 1, \quad \lim_{t \to -\infty} G(t) = 0$
		\item Dichte $g(t) = G'(t)$
		\item $m_1 = \int_{-\infty}^{\infty}t*g(t)dt$
	\end{itemize}
	\subsection{Binominalverteilung}
	\subsubsection{Allgemein}
	\[
		\mathcal{B}(k | p,n) \enspace \textbf{ oder auch } \enspace B(k;p,n) = 
		\begin{pmatrix} n \\ k \end{pmatrix} p^k(1-p)^{n-k} \enspace \newline
		\text{mit k = 0,1,2,...,n} \]
		- wobei diese Funktion die \textbf{kommulierte} Wahrscheinlichkeit angibt, also z.B.
		wobei k = 2 die Wahrscheinlichkeit "1 oder 2"
		\\ - p ist die Wahrscheinlichkeit fuer ein positives Ereigbnis
		\\ - n ist Anzaehl wie oft wir ziehen
	
	\subsubsection{Beispiel: 500 Druckfehler auf 500 Seiten}
		Wie hoch ist die Wahrscheinlichkeit, dass auf einer Seite mindestens 3 Druckfehler
		sind?
		\[
			1- \sum_{k=0}^{2} \mathcal{B}(k|p,n) \enspace mit \enspace \] \\
			k=0,1,2 (Gegenereignisse)\\ n = 500 
			(wir ziehen Fehler "ohne zuruecklegen") \\ p=1/500 (die Wahrscheinlichkeit dass 
			ein Fehler auf einer bestimmten Seite ist)\\
		\begin{equation*}
		\begin{split}
			1- \sum_{k=0}^{2} \mathcal{B}(k|1/500,500)& = 1 - \mathcal{B}(0|1/500,500) - \mathcal{B}(1|1/500,500) - \mathcal{B}(2|
			1/500,500) \\
			& = 1 -  \mathcal{B}(0|1/500,500) - \mathcal{B}(1|1/500,500) - \mathcal{B}(2|
			1/500,500) \\
			& = 1 - \left( \frac{499}{500} \right) ^{500} - 500\frac{1}{500}\left(\frac{499}{500}\right)^{499} - \frac{500*499}{1*2}\left( \frac{1}{500} \right) ^2 \left( \frac{499}{500} \right) ^{498} \\ & = 0,08
		\end{split}
		\end{equation*}
	\subsection{Possion-Verteilung}
	\subsubsection{Allgemein}
	Ereignisse muessen mit konstanter Rate, unabhaengig voneinander und in einem festen 
	Bereich (Modell) stattfinden!
	\[
		P_{\lambda}(n) = \frac{\lambda ^n}{n!} e ^{- \lambda}
	\]
\subsection{N(0,1)-Verteilung}
	$f(x) = \frac{1}{\sqrt{2\pi}}*e^{-0.5x^2}$
\subsection{Normal-Verteilung}
	$f(x) = N(\mu, \sigma^2) = \frac{1}{\sqrt{2\pi\sigma^2}}*e^{-\frac{1}{2\sigma^2}(x-		
	\mu)^2} \quad \quad m_1 = \mu \quad \quad \widehat{m}_2=\sigma^2$
\subsection{Exponentiallverteilung}
	$f(\lambda) = \lambda*e^{-\lambda t}$

\subsection{Laplace-Verteilung}
	Zufallsexperimente, bei denen jedes Ergebnis die gleiche Chance hat. \\
	$f(w) = L(\Omega) = \frac{1}{|\Omega|}$
\subsection{Hypergeometrische Verteilung}
	Zufallsexperimente, bei denen man die Ergebnisse als Anzahlen von schwarzen Kugeln unter 	n gezogenen interpretieren kann. \\
	$f(k) = H(N, K, n) = \frac{\binom{K}{k}*\binom{N-K}{n-k}}{\binom{N}{n}}$
\subsection{Geometrische Verteilung}
	Die geometrische Verteilung beschreibt die Wartezeit für das erstmalige Eintreten eines 	
	Ereignisses unter der Annahme der Gedaechtnislosigkeit. \\
	$G(p) = f(n) = p*q^{n-1} \quad \quad m_1 = \frac{1}{p}$
\section{Zufallsvarriablen}
\subsection{Verteilungen von Zufallsvariablen}
	Wir benoetigen mehrdimensionale integration, d.h. wir \textbf{muessen} wissen von
	wo bis wo wir integrieren wollen \\ \\
	\subsubsection{Beispiel, Ereignis gegeben + Verteilungsfunktion gegeben:}
	$Ereignis: \: X_2 > 2X_1 \: => \: \underbrace{\int_{-\infty}^{+\infty}}_{X_1}
	\underbrace{\int_{2X_1}^{+\infty}}_{X_2}$\\ \\ \\
	$Verteilung: \: expotentiell \: => \: f(\lambda) = \lambda e^{-\lambda t}$ \\
	$Ausserdem \: sei: \:  \lambda_1 = 1 \:\, und \:\, \lambda_2 =2 $\\ \\
	Wir integrieren zunaechst ueber $X_2$ d.h. wir sezten $\lambda = 2$
	\[
		\begin{split}
			\int_{-\infty}^{+\infty} \int_{2X_1}^{+\infty} 2 e^{-2 X_2} \: dX_2 dX_1
		\end{split}
	\]
	Fuer $X_1$ setzen wir dann dementsprechend  $\lambda e^{-\lambda t}$ mit $\lambda = 1$ ein, dann nur noch das 2te Integral ausrechnen.
	% \\ \\ \textbf{Moeglichkeit b) - Nach $x_1$ oder $x_2$ umstellen} \\ (ggf. mit Koordinatentransformation)
	\subsubsection{Alternatives Beispiel:}
	X,Y stochastisch unabhaengige, mit Parameter 'p' geometrisch verteilte Zufallsvarriablen 	in Wahrscheinlichkeitsraum $(\varOmega , \mathcal{A},P)$. Welche Verteilung besitzt 
Zufallsvarriable $Z = min(X,Y)$, definiert durch $Z( \omega ) = min \{X(\omega),Y(\omega)\}$.\\


\section{Marginaldichte - Beispielrechnung}
\[	
f(x_z,x_2)=
	\begin{cases}
		ce^{-(2x_1+3x_2)} & x_1  > 0 \: und \: 0 < x_2 <x_1 \\
		0 & sonst
	\end{cases}
\]
Marginaldichte:
\[
	\begin{split}
		f_1(x_1) 	& = \int_{0}^{x_1} f(x_1,x_2) dx_2 \\
					& = \int_{0}^{x_1} ce^{-2x_1}e^{-3x_2} \: dx_2 \\
					& = \underbrace{c*e^{-2x_1}}_{\text{Konstante, da Integration nach} \: x_2}  \overbrace{\int_{0}^{x_1} e^{-3x_2} \: dx_2}^{mit \: 0 \: und \: x_1 \: einsetzen \: integrieren} \\
					& = ce^{-2x_1}\, \frac{1}{3} (1-e^{-3x_2} )
	\end{split}
\]
Gegebenenfalls koennen wir das gleiche auch mit $dx_2$ tun wenn das einfach zur integrieren ist oder nach beiden Marginaldichten gefragt ist.
Damit $f(x_1,x_2)$ und $f_2(x_2)$ Dichten sind muss gelten:
\[
	\int f_2(x_2) dx_2 = 1
\]
und:
\[
	\int \int f(x_1,x_2) dx_1 dx_2 = 1
\]
\section{Koordinatentransformation}
\section{Komposition von Zufallsvektoren}
\begin{center}
\begin{tikzpicture}[
	  bend angle=45,
	  scale = 1.5,
	  pre/.style={<-,shorten <=1pt,>=stealth',semithick},
	  post/.style={->,shorten >=1pt,>=stealth',semithick},
	  mid/.style={-,shorten >=1pt,>=stealth',semithick},
	  place/.style={circle,draw=red!50,fill=red!20,thick}]

	  \node[place]  (A) at ( 0,0)[label=above:Before] {$(\Omega, A, P) $};
	  \node[place]  (B) at ( 2,0) {$(R^n, B_n, P^X)$}
		edge [pre] node [auto] {X} (A);
		\node[place, align=center]  (C) at ( 2,-3) {$(R^m, B_m, P^G$}
		edge [pre] node [auto] {$Y = G \circ X$} (A)
		edge [pre] node [auto] {$G$} (B);
\end{tikzpicture}
\end{center}
\vspace*{7pt}
\textbf{Gegeben:} Man hat stochastisch unabhängige Zufallsvariablen $X_1, \ldots , X_n$ gegeben mit
Art der Verteilung.

\textbf{Gesucht:} Verteilung von Zufallsvariable $Y$, die sich aus $X_i$ berechnen lässt.

\textbf{Beispiel:}\\
Welche Verteilung besitzt
\begin{align}
	X = \frac{X_1}{X_1 + X_2}
\end{align}
falls $X_1$ und $X_2$ exponentiell verteilt mit Paramter $\lambda$ und stochastisch
unabhängig sind.

\begin{enumerate}
	\item Wegen Unabhängigkeit der Variablen $X_1$ und $X_2$ besitzt $P^X$
		die Dichte $f(x_1,x_2) = f_1(x_1)f_2(x_2)$.
	\item $M = {(x_1, x_2); x_1 > 0 \text{ und } x_2 > 0}$\\
		$\longrightarrow$ Wertebereich von $x_n$ anhand von Verteilung ermitteln.
	\item Gleichungen  $G(x)$ definieren:
		\begin{align}
			y_1 = \frac{x_1}{x_1 + x_2}\\
			y_2 = x_2
		\end{align}
	\item Funktionaldeterminante ($J_{G(x)}$) der Abbildung $G$ berechnen
		\begin{align}
			J_{G}(x) =
			\text{det} \begin{pmatrix}
				\frac{\partial G_1}{\partial x_1} (x) & \cdots & \frac{\partial G_1}{\partial x_n} (x) \\
				\vdots  & \ddots & \vdots  \\
				\frac{\partial G_n}{\partial x_1} (x) & \cdots & \frac{\partial G_n}{\partial x_n} (x) \\
			\end{pmatrix}\\
			J_{G}(x_1,x_2) =
			\text{det} \begin{pmatrix}
				\frac{x_2}{(x_1 + x_2)^2} & * \\
					0 & 1 \\
			\end{pmatrix} = \frac{x_2}{(x_1 + x_2)^2}
		\end{align}
	\item Umkehrabbildung $G^*$ berechnen. Alle Zufallsvariablen werden
		werden mittels Funktionen verändert: z.B: $y_1 = x_1/x_2$.
		Jede i-te Funktion nach $x_i$ auflösen.
		\begin{align}
			x1 = \frac{y_1y_2}{1 - y_1}\\
			x_2 = y_2
		\end{align}
	\item Gesuchte Funktion: $g(y) = f(G^*(y))\frac{1}{|J_G(G^*(y))|}$\\
		$\longrightarrow$ Setze für alle $x_i$ dementsprechend $y_i$ ein und multipliziere
		mit Kehrwehrt von Funktionaldeterminante.
		\begin{align}
			g(y_1,y_2) = \lambda^2e^{-\frac{\lambda}{1 - y_1}}\frac{y_2}{(1-y_1)^2}
		\end{align}
	\item Mit Marginaldichte $g_1(y_1)$ berechnen:\\
		\begin{align}
			g_1(y_1) = \frac{\lambda}{1 - y_1} \int^\infty_0 y_2\frac{\lambda}{1 - y_1}
			e^{-\frac{\lambda}{1 - y_1}} dy_2\\
			= \frac{\lambda}{1 - y_1} m_1 (\varepsilon(\frac{\lambda}{1 - y_1}))\\
			= 1
		\end{align}
		$\longrightarrow$ Da Mittelwert der $\varepsilon$-Verteilung gerade Kehrwert des
		Paramters ist.
	\item Folgerung: Dichte $g_1$ ist also die der Uniform-Verteilung ($U(0,1)$).
\end{enumerate}

\end{document}
